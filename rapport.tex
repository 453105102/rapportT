%%%%%%%%%%%%%%%%%%%%%%%%%%%%%%%%%%%%%%%%%%%%%%%%%%%%%%%%%%%%%%%%%%%%
%           Auteur :  P.TRAN BA, E.BOUTTIER         %
%         Création :  08/06/2012 18:05                             %
%%%%%%%%%%%%%%%%%%%%%%%%%%%%%%%%%%%%%%%%%%%%%%%%%%%%%%%%%%%%%%%%%%%%

\documentclass[a4paper,11pt]{article}

% \usepackage[utf8]{inputenc}
% \usepackage[T1]{fontenc}
% %\usepackage{xunicode}
% \usepackage{fontspec}
% \defaultfontfeatures{Mapping=tex-text,Scale=MatchLowercase}
% \usepackage{a4wide}
% \usepackage{verbatim}
% %\usepackage{polyglossia}
% %\setdefaultlanguage{french}
% %~ \usepackage{listings}
% \usepackage[french]{babel}
% %~ \usepackage{url}
% %~ \usepackage{times}
% %\usepackage{minted}
% \usepackage{graphicx}
% %\input{graphviz}

\usepackage{pack_roman}

\usepackage{geometry}
\geometry{hmargin=2.5cm,vmargin=1.5cm}

\title{Projet de Traitement du Signal\\Segmentation d'image SAR\\Rapport}
\author{P.TRAN BA, E.BOUTTIER}
\date\today

\begin{document}

\maketitle

\begin{abstract}

This report deals with a way to detect the edge using the normalized Ratio Of Exponentially Weighted Average (ROEWA) on opposite sides of the central pixel. It is use in both direction horizontal and vertical, and the magnitude of the two components yields an edge strength map.

This edge detector can cope with the presence of speckle, which can be modeled as a strong multiplicative noise. This report describe step by step the processus used in order to simulate a segmentation with MatLab.

\end{abstract}

\tableofcontents

\newpage

\section{Introduction}
\subsection{Préface}

En analyse d'images, la segmentation est une étape essentielle, préliminaire a des traitements de haut niveau tels que la classification, la détection ou l'extraction d'objets. Elle consiste à décomposer une image en régions homogènes. Les deux principales approches sont l'approche région et l'approche contour. L'approche région cherche à regrouper les pixels présentant des propriétés communes tandis que l'approche contour vise à détecter les transitions entre régions. Des détecteurs efficaces ont été développés dans le cadre de l'imagerie optique, mais s'avèrent inadaptés aux images radar de par la présence d'un bruit multiplicatif appelé speckle.

\subsection{Objectif du projet}

L'objectif de ce projet est d'effectuer la segmentation d'une image radar a synthèse d'ouverture (Image RSO ou Image SAR pour Synthetic Aperture Radar) à l'aide d'une méthode originale de détection de
ruptures appliquée successivement sur les lignes et colonnes de l'image. La méthode
est issue d'une publication intitulée “An Optimal Multiedge Detector for SAR Image
Segmentation” publiée en mai 1998 dans la revue IEEE Transactions on Geoscience
and Remote Sensing

\section{Génération d'une ligne d'image SAR}

Cette partie consiste a générer des lignes d'image radar conformément au modèle proposée par l'article. La méthode de segmentation choisie sera d'abord testée sur ces lignes
avant d'être appliquée a des images entières.

\subsection{Ligne d'image non bruitée R(x)}



\end{document}
