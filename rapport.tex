%%%%%%%%%%%%%%%%%%%%%%%%%%%%%%%%%%%%%%%%%%%%%%%%%%%%%%%%%%%%%%%%%%%%
%           Auteur :  P.TRAN BA, E.BOUTTIER         %
%         Création :  08/06/2012 18:05                             %
%%%%%%%%%%%%%%%%%%%%%%%%%%%%%%%%%%%%%%%%%%%%%%%%%%%%%%%%%%%%%%%%%%%%

\documentclass[a4paper,11pt]{article}

% \usepackage[utf8]{inputenc}
% \usepackage[T1]{fontenc}
% %\usepackage{xunicode}
% \usepackage{fontspec}
% \defaultfontfeatures{Mapping=tex-text,Scale=MatchLowercase}
% \usepackage{a4wide}
% \usepackage{verbatim}
% %\usepackage{polyglossia}
% %\setdefaultlanguage{french}
% %~ \usepackage{listings}
% \usepackage[french]{babel}
% %~ \usepackage{url}
% %~ \usepackage{times}
% %\usepackage{minted}
% \usepackage{graphicx}
% %\input{graphviz}

\usepackage{pack_roman}

\usepackage{geometry}
\geometry{hmargin=2.5cm,vmargin=1.5cm}

\title{Projet de Traitement du Signal\\Segmentation d'image SAR\\Rapport}
\author{P.TRAN BA, E.BOUTTIER}
\date\today

\begin{document}

\maketitle

\begin{abstract}

This report deals with a way to detect the edge using the normalized Ratio Of Exponentially Weighted Average (ROEWA) on opposite sides of the central pixel. It is use in both direction horizontal and vertical, and the magnitude of the two components yields an edge strength map.

This edge detector can cope with the presence of speckle, which can be modeled as a strong multiplicative noise. This report describe step by step the processus used in order to simulate a segmentation with MatLab.

\end{abstract}

\tableofcontents

\newpage

\section{Introduction}
\subsection{Préface}



\subsection{Rappel du sujet}



\end{document}
